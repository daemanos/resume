% !TEX TS-program = xelatex
% !TEX encoding = UTF-8 Unicode
% -*- coding: UTF-8; -*-
% vim: set fenc=utf-8

%%%%%%%%%%%%%%%%%%%%%%%%%%%%%%%%%%%%%%%%%%%%%%%%%%%%%%%%%%%%%%%%%
%% CV.tex
%% <https://github.com/zachscrivena/simple-resume-cv>
%% This is free and unencumbered software released into the
%% public domain; see <http://unlicense.org> for details.
%%%%%%%%%%%%%%%%%%%%%%%%%%%%%%%%%%%%%%%%%%%%%%%%%%%%%%%%%%%%%%%%%

\documentclass[letterpaper,MMMyyyy,nonstopmode]{simpleresumecv}
% Class options:
% a4paper, letterpaper, nonstopmode, draftmode
% MMMyyyy, ddMMMyyyy, MMMMyyyy, ddMMMMyyyy, yyyyMMdd, yyyyMM, yyyy

%%%%%%%%%%%%%%%%%%%%%%%%%%%%%%%%%%%%%%%%%%%%%%%%%%%%%%%%%%%%%%%%%
%% PREAMBLE.
%%%%%%%%%%%%%%%%%%%%%%%%%%%%%%%%%%%%%%%%%%%%%%%%%%%%%%%%%%%%%%%%%

% CV Info (to be customized).
\newcommand{\CVAuthor}{Daman Morris}
\newcommand{\CVTitle}{Daman Morris's Resume}
\newcommand{\CVNote}{}
\newcommand{\CVWebpage}{daeman.me}

% PDF settings and properties.
\hypersetup{%
pdftitle={\CVTitle},
pdfauthor={\CVAuthor},
pdfsubject={},
pdfcreator={XeLaTeX},
pdfproducer={},
pdfkeywords={},
unicode=true,
bookmarks=true,
bookmarksopen=true,
pdfstartview=FitH,
pdfpagelayout=OneColumn,
pdfpagemode=UseOutlines,
hidelinks,
breaklinks}

% Shorthand.
\newcommand{\Code}[1]{\mbox{\textbf{#1}}}
\newcommand{\CodeCommand}[1]{\mbox{\textbf{\textbackslash{#1}}}}

%%%%%%%%%%%%%%%%%%%%%%%%%%%%%%%%%%%%%%%%%%%%%%%%%%%%%%%%%%%%%%%%%
%% ACTUAL DOCUMENT.
%%%%%%%%%%%%%%%%%%%%%%%%%%%%%%%%%%%%%%%%%%%%%%%%%%%%%%%%%%%%%%%%%

\begin{document}

%%%%%%%%%%%%%%%
% TITLE BLOCK %
%%%%%%%%%%%%%%%

\Title{\CVAuthor}

\begin{SubTitle}
397 Parkside Avenue, Pittsburgh, PA 15228
\,\SubBulletSymbol\,
1085 Nathaniel Rochester Hall, Rochester, NY 14623
\par
\href{mailto:damanm72@gmail.com}
{damanm72@gmail.com}
\,\SubBulletSymbol\,
+1\,(412)\,586-8168
\,\SubBulletSymbol\,
\href{\CVWebpage}
{\url{\CVWebpage}}
\end{SubTitle}

\begin{Body}

%%%%%%%%%%%%%%%
%% EDUCATION %%
%%%%%%%%%%%%%%%

\Section
{Education}
{Education}
{PDF:Education}

\Entry
\href{http://www.rit.edu}
{\textbf{Rochester Institute of Technology}},
Rochester, NY

\Gap
\BulletItem
B.S. in
\href{http://cs.rit.edu}
{Computer Science}
\hfill
\DatestampYMD{2016}{08}{22} --
Present
\begin{Detail}
\SubBulletItem
Classes:
Found.\ of Intelligent Systems, Intro.\ to Computer Vision, \\
Science \& Analytics of Speech, Game Theory
\SubBulletItem
Cumulative GPA: 3.41 / 4.0
\end{Detail}

%%%%%%%%%%%%%%%%
%% EXPERIENCE %%
%%%%%%%%%%%%%%%%

\Section
{Experience}
{Experience}
{PDF:Experience}

\Entry
\href{http://cs.rit.edu}
{\textbf{RIT CS Department}},
Rochester, NY

\Gap
\BulletItem
Grader,
Concepts of Computer Systems
\hfill
\DatestampYMD{2018}{01}{18} --
Present
\begin{Detail}
\SubBulletItem
Graded tests involving low-level computer organization and assembly language.
\SubBulletItem
Learned valuable time management skills.
\SubBulletItem
Maintained consistency of scoring across a section of tests.
\end{Detail}

\BigGap
\Entry
\textbf{A Meal \& More, Inc.},
Rochester, NY

\Gap
\BulletItem
Volunteer
\hfill
Spring 2017
\begin{Detail}
    \SubBulletItem
    Helped prepare and serve food to those in need in Rochester.
\end{Detail}

%%%%%%%%%%%%
%% SKILLS %%
%%%%%%%%%%%%

\Section
{Skills}
{Skills}
{PDF:Skills}

\BulletItem
Programming languages:
Python,
Lua,
C,
Java,
Rust,
Julia
(\LaTeX, HTML, CSS)

\BulletItem
Disciplines:
Computer Vision \& Image Processing,
Functional Programming, % iffy
Linguistics,
Machine Learning,
Data Mining \& Analysis,
Systems Programming,
Database Programming

\BulletItem
Technologies:
OpenCV,
Pandoc,
H2,
SQL,
Inkscape,
GNU Image Program

%%%%%%%%%%%%%%
%% PROJECTS %%
%%%%%%%%%%%%%%

\Section
{Projects}
{Projects}
{PDF:Projects}

\Entry
\textbf{Jet Database Engine}

\BulletItem
General-purpose library for rapid development of small database applications
in pure Java.

\Gap
\BulletItem
Supports a wide variety of drivers that work with the Java SQL framework.

\BigGap
\Entry
\textbf{Pangloss}

\BulletItem
Pandoc filter for interlinear glosses (a type of example used in grammars and
other linguistic documents with a sentence, its constituent parts, and an
English translation on separate lines).

\Gap
\BulletItem
Allows pandoc example lists to be used to generate interlinear glosses in PDF
with the \LaTeX{} gb4e package or HTML with Leipzig.js. Written in Python.

\BigGap
\Entry
\textbf{Lexis}

\BulletItem
Standalone tool to facilitate the creation of professionally typeset
dictionaries with Pandoc and markdown.

\Gap
\BulletItem
Allows dictionaries to be specified as Markdown lists, either in one file or
multiple files by letter, and supports arbitrary collation orders for different
languages. Written in Python.

%%%%%%%%%%%%%%%%%%%%%%%%%%%
%% AWARDS & SCHOLARSHIPS %%
%%%%%%%%%%%%%%%%%%%%%%%%%%%

\Section
{Awards \&\newline
Scholarships}
{Awards \& Scholarships}
{PDF:AwardsAndScholarships}

\BulletItem
Dean's List,
Spring 2017 through Fall 2018,
Rochester Institute of Technology
\hfill
\DatestampY{2017} --
\DatestampY{2018}
\begin{Detail}
\Item
For attaining a semester GPA of at least 3.4 with no grades of `D', `F', or
incomplete.
\end{Detail}

\Gap
\BulletItem
Presidential Scholar Award,
Rochester Institute of Technology
\hfill
\DatestampYMD{2016}{01}{15}
\begin{Detail}
\Item
For exceptional academic performance and strong entrance exam scores.
\end{Detail}

%%%%%%%%%%%%%%%%%%%%%%%
%% CAMPUS ACTIVITIES %%
%%%%%%%%%%%%%%%%%%%%%%%

\Section
{Campus Activities}
{Campus Activities}
{PDF:CampusActivities}

\Entry
{\textbf{PiRIT}},
Rochester Institute of Technology

\Gap
\BulletItem
Member
\hfill
\DatestampYMD{2018}{01}{19} --
Present

%%%%%%%%%%%%%%%
%% LANGUAGES %%
%%%%%%%%%%%%%%%

\Section
{Languages}
{Languages}
{PDF:Languages}

\BulletItem
English: Native language.

\Gap
\BulletItem
French: Intermediate (reading, writing); basic (speaking).

\Gap
\BulletItem
Portuguese: Basic (reading, writing, speaking).

%%%%%%%%%%%%%%%
%% INTERESTS %%
%%%%%%%%%%%%%%%

\Section
{Interests}
{Interests}
{PDF:Interests}

\Entry
Conlanging,
reading,
typography,
graphic design.

%%%%%%%%%%%%%%%%
%% REFERENCES %%
%%%%%%%%%%%%%%%%

%\Section
%{References}
%{References}
%{PDF:References}

%\BulletItem
%\textbf{Professor Jonathan Public}
%\newline
%Professor of Geology and Mechanical Engineering
%\newline
%First American University
%\newline
%1000 First Avenue, Springfield, Massachusetts 22222, USA
%\newline
%\href{mailto:jonathanpublic@example.com}
%{jonathanpublic@example.com}
%\,\SubBulletSymbol\,
%+1\,(555)\,222-2222

%\BigGap
%\BulletItem
%\textbf{Dr Alice Bob Carol}
%\newline
%Director, Research \& Development
%\newline
%Alpha Engineering Firm
%\newline
%20 North Street, Oakland, Ohio 33333, USA
%\newline
%\href{mailto:alicebobcarol@example.com}
%{alicebobcarol@example.com}
%\,\SubBulletSymbol\,
%+1\,(555)\,333-3333

%%%%%%%%%%%%%%%%%%%%%%%%%%%%%%%%%%%%%%%%
%% THIS IS A SECTION WITH USAGE NOTES %%
%%%%%%%%%%%%%%%%%%%%%%%%%%%%%%%%%%%%%%%%

%% Declare a new group to limit the scope of \color to this section.
%\begingroup
%\color{red}

%\Section
%{This is a\newline
%Section\newline
%With\newline
%Usage Notes}
%{This is a Section With Usage Notes (For PDF Bookmark)}
%{PDF:ThisIsASectionWithUsageNotes:ForPDFLink}

%\SubSection
%{This is a SubSection}
%{This is a SubSection (For PDF Bookmark)}
%{PDF:ThisIsASubSection:ForPDFLink}

%\BigGap
%\BulletItem
%Use \CodeCommand{Section\{a\}\{b\}\{c\}} and
%\CodeCommand{SubSection\{a\}\{b\}\{c\}}
%to create sections and subsections, where
%\Code{a} is the heading displayed on the page,
%\Code{b} is the PDF bookmark heading, and
%\Code{c} is the internal PDF link (must be unique).
%Sections and subsections will appear in the PDF bookmarks.
%Note the CamelCase command names.

%\Gap
%\BulletItem
%Use
%\CodeCommand{Entry},
%\CodeCommand{BulletItem},
%\CodeCommand{SubBulletItem},
%\CodeCommand{Item},
%\CodeCommand{SubItem},
%\CodeCommand{NumberedItem},
%etc.,
%to create entries in the main body of the CV.

%\Gap
%\BulletItem
%Enclose entry details between
%\CodeCommand{begin\{Detail\}} and
%\CodeCommand{end\{Detail\}}
%so that they are typeset in a smaller font.
%\begin{Detail}
%\Item
%This is an example of entry detail text enclosed in a \Code{Detail} environment.
%\end{Detail}

%\Gap
%\BulletItem
%Use \CodeCommand{Gap} and \CodeCommand{BigGap} to insert vertical spaces between entries to improve layout.

%\BigGap
%\SubSection
%{This is Another SubSection}
%{This is Another Subsection (For PDF Bookmark)}
%{PDF:ThisIsAnotherSubSection:ForPDFLink}

%\BigGap
%\Entry
%This is a plain \CodeCommand{Entry},
%followed by an \CodeCommand{hfill} and a date range
%\hfill
%\DatestampYM{2015}{10} --
%\DatestampYM{2015}{12}

%\Gap
%\BulletItem
%This is a \CodeCommand{BulletItem}.
%\Item
%This is an \CodeCommand{Item}, which has no bullet.
%Note the alignment with the \CodeCommand{BulletItem} above.

%\Gap
%\SubBulletItem
%This is a \CodeCommand{SubBulletItem}.
%\SubItem
%This is a \CodeCommand{SubItem}, which has no bullet.
%Note the alignment with the \CodeCommand{SubBulletItem} above.

%\Gap
%\NumberedItem{[42]}
%This is a \CodeCommand{NumberedItem}.
%Change the value of the macro \CodeCommand{MaxNumberedItem} to adjust the indentation width.

%\BigGap
%\SubSection
%{Line, Paragraph, and Page Breaks}
%{Line, Paragraph, and Page Breaks (For PDF Bookmark)}
%{PDF:LineParagraphAndPageBreaks:ForPDFLink}

%\BigGap
%\BulletItem
%To create a new line within the same paragraph (i.e., preserving the same paragraph indentation), use \CodeCommand{newline} instead of \CodeCommand{\textbackslash};
%the latter will reset the paragraph indentation.

%\Gap
%\BulletItem
%To create a new paragraph, use \CodeCommand{par} or simply leave an empty line.
%Paragraph indentations (from
%\CodeCommand{Entry},
%\CodeCommand{BulletItem},
%\CodeCommand{SubBulletItem},
%\CodeCommand{Item},
%\CodeCommand{SubItem},
%\CodeCommand{NumberedItem},
%etc.) do not carry across different paragraphs.

%\Gap
%\BulletItem
%To create a new page, use \CodeCommand{newpage}.

%\BigGap
%\SubSection
%{Dates}
%{Dates (For PDF Bookmark)}
%{PDF:Dates:ForPDFLink}

%\BigGap
%\BulletItem
%Use the following macros to specify and display dates consistently:
%\SubBulletItem
%\CodeCommand{DatestampYMD\{yyyy\}\{MM\}\{dd\}}
%(e.g., \CodeCommand{DatestampYMD\{2008\}\{01\}\{15\}})
%\SubBulletItem
%\CodeCommand{DatestampYM\{yyyy\}\{MM\}}
%(e.g., \CodeCommand{DatestampYM\{2008\}\{01\}})
%\SubBulletItem
%\CodeCommand{DatestampY\{yyyy\}}
%(e.g., \CodeCommand{DatestampY\{2008\}})

%\Gap
%\BulletItem
%Change the date format option passed to the document class to adjust how dates are displayed throughout the document:
%\SubBulletItem
%\Code{MMMyyyy} (``Jan~2008'')
%\SubBulletItem
%\Code{ddMMMyyyy} (``15~Jan~2008'')
%\SubBulletItem
%\Code{MMMMyyyy} (``January~2008'')
%\SubBulletItem
%\Code{ddMMMMyyyy} (``15~January~2008'')
%\SubBulletItem
%\Code{yyyyMMdd} (``2008-01-15'')
%\SubBulletItem
%\Code{yyyyMM} (``2008-01'')
%\SubBulletItem
%\Code{yyyy} (``2008'')

%\endgroup

%%%%%%%%%%%%%%%%%%%%%%%%%
%% RESEARCH EXPERIENCE %%
%%%%%%%%%%%%%%%%%%%%%%%%%

%\Section
%{Research Experience}
%{Research Experience}
%{PDF:ResearchExperience}

%\Entry
%\href{http://www.example.com/my-institute}
%{\textbf{Institute for Advanced Research}},
%Science College

%\Gap
%\BulletItem
%Undergraduate Research Student, Science Department
%\hfill
%\DatestampYMD{2004}{05}{15} --
%\DatestampYMD{2005}{05}{15}
%\begin{Detail}
%\SubBulletItem
%Project:
%Investigations on the Use of Lasers to Measure Climate Change
%\SubBulletItem
%Supervisors:
%Prof.~Jane~Citizen and
%Dr~Ann~Yone
%\SubBulletItem
%Focus:
%Climate change, lasers, statistical analysis, data analytics.
%\end{Detail}

%%%%%%%%%%%%%%%%%%
%% PUBLICATIONS %%
%%%%%%%%%%%%%%%%%%

%\Section
%{Publications}
%{Publications}
%{PDF:Publications}

%\SubSection
%{Journals}
%{Journals}
%{PDF:Journals}

%% Declare a new group to limit the scope of \MaxNumberedItem to this subsection.
%\begingroup
%\renewcommand{\MaxNumberedItem}{[88]}

%\BigGap
%\NumberedItem{[10]}
%\href{http://www.example.com/my-paper-doi-5}
%{\underline{J.~Doe}, J.~Citizen, and A.~Yone,
%``On lasers and climate change,''
%\textit{Journal of Science},
%vol.~89,
%no.~2,
%pp.~4123--4133,
%\DatestampYM{2008}{02}.}

%\Gap
%\NumberedItem{[1]}
%\href{http://www.example.com/my-paper-doi-4}
%{\underline{J.~Doe} and J.~Citizen,
%``Measuring the extent of climate change,''
%\textit{Global Scientific Journal},
%vol.~12,
%no.~4,
%pp.~330--352,
%\DatestampYM{2006}{12}.}

%\endgroup

%\BigGap
%\SubSection
%{Conferences}
%{Conferences}
%{PDF:Conferences}

%% Declare a new group to limit the scope of \MaxNumberedItem to this subsection.
%\begingroup
%\renewcommand{\MaxNumberedItem}{[8888]}

%\BigGap
%\NumberedItem{[1000]}
%\href{http://www.example.com/my-paper-doi-3}
%{\underline{J.~Doe}, J.~Citizen, and A.~Yone,
%``On lasers and climate change,''
%in \textit{Proceedings of the Laser Symposium},
%Las Vegas, Nevada, USA,
%\DatestampYM{2007}{01}.}

%\Gap
%\NumberedItem{[100]}
%\href{http://www.example.com/my-paper-doi-2}
%{A.~Yone and \underline{J.~Doe},
%``Climate change and general relativity,''
%in \textit{Proceedings of the International Astronomical Conference},
%Sydney, Australia,
%\DatestampYM{2006}{8}.}

%\Gap
%\NumberedItem{[10]}
%\href{http://www.example.com/my-paper-doi-1}
%{\underline{J.~Doe} and J.~Citizen,
%``Measuring the extent of climate change,''
%in \textit{Proceedings of the International Climate Change Conference},
%London, UK,
%\DatestampYM{2005}{11}.}

%\endgroup

\end{Body}

%%%%%%%%%%%
% CV NOTE %
%%%%%%%%%%%

%\BigGap
%\UseNoteFont%
%\null\hfill%
%[\textit{\CVNote}]

\end{document}
